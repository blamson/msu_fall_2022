\documentclass[11pt]{article}      % Specifies the document class
\parskip=5pt
%% amount of indent you want before the first line of each paragraph
\parindent=0pt
%% Uncomment the next line if you want to use Adobe Times postscript fonts
%\usepackage{times}
%% Uncomment the next line if you want to use AMS-LaTeX package
%\usepackage{amsmath}
%\numberwithin{equation}{section} %% {part of the amsmath package}

\renewcommand{\baselinestretch}{1.1}

\addtolength{\oddsidemargin}{-.6in}
\addtolength{\evensidemargin}{-.6in}
\addtolength{\textwidth}{1.75in}

\addtolength{\topmargin}{-.875in}
\addtolength{\textheight}{1in}

\usepackage{amsmath}
\usepackage{amsthm}
\usepackage{mdframed}
\usepackage{multicol}
\usepackage{enumitem}
\usepackage{fancyhdr}
\usepackage[title]{appendix}
\usepackage{systeme,mathtools}

\usepackage[utf8]{inputenc}

\title{Assignment 2}
\author{lamsonbrady }
\date{September 3rd, 2022}

% General packages
\usepackage{parskip}

% Header related stuff
\usepackage{fancyhdr}
\pagestyle{fancy}
\fancyhead[R]{MTH 3220 - Linear Algebra}  
\fancyhead[L]{Homework 2}               
\fancyfoot[R]{Brady Lamson}
\fancyfoot[L]{MSU Denver}

% Math packages
\usepackage{amsmath}
\usepackage{amssymb}

% Creates small p-matrix ----
\newcommand{\psmall}[1]{
\left(\begin{smallmatrix}
#1
\end{smallmatrix} \right)
}

% makes mathbb commands faster to type
\newcommand{\bb}{\mathbb}

% Makes vectors slightly faster
% \newcommand{\v}[#1]{\vec{#1}}


\renewcommand{\baselinestretch}{1.1}



\newcommand{\mz}{\mathbf{Z}}
\newcommand{\mq}{\mathbf{Q}}
\newcommand{\mr}{\mathbf{R}}
\newcommand{\mc}{\mathbf{C}}
\newcommand{\lra}{\longrightarrow}



\newmdtheoremenv{theorem}{Theorem}
\newmdtheoremenv{lemma}[theorem]{Lemma}
\newmdtheoremenv{proposition}[theorem]{Proposition}
\newmdtheoremenv{corollary}[theorem]{Corollary}
\newmdtheoremenv{mydef}[theorem]{Definition}
\newmdtheoremenv{example}[theorem]{Example}


\usepackage{multicol}

\usepackage{fancyhdr}

\pagestyle{fancy} \fancyhead[RO, RE] {\texttt{ The Inverse of a Matrix}}

\begin{document}

The starting idea for this section, is to understand the difference between linear equations of the form $ax=b$ and matrix equations of the form $AX=B$:

$ax = b$

$a = b/a$ IF $a \neq 0$.

For matrices this is slightly different as they lack division. Go back to the above equation, we can rewrite it as

$a = b(\frac{1}{a}) = b \cdot a^{-1}$

We simply multiply b by the inverse of a. That same principle can be used for matrices! Remember, division doesn't actually exist, it's just multiplication times an inverse.

\[A^{-1} \cdot (AX = B)\]

If $A^{-1} \cdot A = 1$, then 
\[A^{-1} \cdot (AX) = A^{-1}B\] 

\[x = A^{-1}B\]

\hrule
\begin{center}
\underline{I. The Identity Matrix $\mathbf{I_n}$}
\end{center}
\textbf{Step 1}:  What does it mean that a matrix $A$ has an inverse? 

A has an inverse if I can reverse multiplying by A.

\textbf{Note!} Order of multiplication matters here, we need $AA^{-1} = 1$

\vskip .5truein
\textbf{Step 2}:  What is the equivalent of "1" in the matrix universe?

One is $1 \cdot X = X$ for any matrix.
\vskip .5truein
\hrule
\pagebreak
\begin{mydef}
The identity matrix $\mathbf{I_n}$ is the matrix:

$$\mathbf{I_n}=\left( \begin{array}{ccccc}
1 & 0 &0 &&\\
0 & 1 &0 &&\\
0 & 0 & 1&&\end{array} \right)$$

So 

\[
	I_n = \left( \vec{e_1}, \vec{e_2}, \vec{e_n} \right)
\]

\end{mydef}
Examples:

Recall matrix multiplication

\[
A + 0n = A \\
\genpmat{1&2\\3&4} + \genpmat{0&0&0&0} = \genpmat{1&2\\3&4}
\]

Matrix Multiplication

\[A \cdot I_n = A\]

\[\genpmat{1&2\\3&4} \cdot \genpmat{1&0&0&1} = \genpmat{1&2\\3&4}\]

\textbf{Notes on Identity Matrix}

\[I_2 = \genpmat{1&0\\0&1}\]

\[I_3 = \genpmat{1&0&0\\0&1&0\\0&0&1}\]

Note that only square matrices might have an inverse.

The identity matrix is always a square matrix.

\pagebreak
\begin{center}
\underline{II. The Inverse Matrix $A^{-1}$}
\end{center}

\begin{mydef}
Given a square $n\times n$ matrix $A$, we say that $A$ has an inverse, $A^{-1}$ if:
$$AA^{-1}=A^{-1}A=\mathbf{I_n}$$
\end{mydef}

\begin{example}
Consider the matrix we talked about last time, $A=\left( \begin{array}{cc}
-1 &0\\
0 & -1 \\
\end{array} \right)$. Recall that this represents a linear transformation that: 

\[T\genpmat{x\\y} = \genpmat{-1&0\\0&-1}\genpmat{x\\y} = \genpmat{-x\\-y}\]

What would the inverse matrix mean in the geometric description? What is the inverse here?
\end{example}

Here $A \cdot A = I_2$

\[
	\genpmat{-1&0\\0&-1}
	\genpmat{-1&0\\0&-1}
	=
	\genpmat{1&0\\0&1}
\]


\vskip 1truein

\begin{example}
Find another matrix $A$ such that: $A^2 = I_2$
\end{example}

For example, A = symmetry over y-axis:

\[A\vec{v} = A\genpmat{x\\y} = \genpmat{-x\\y}, A = \genpmat{-1&0&0&1}\]

Then $A^2 = I_2$

\pagebreak

\begin{center}
\underline{III. An Algebraic Method to Find the Inverse Matrix $A^{-1}$}
\end{center}

Recall that, by definition, we need to solve the equation $A A^{-1}=\mathbf{I_n}$:

\[A \cdot A^{-1} = \genpmat{1&0&0\\0&1&0\\0&0&1}\]

\[A(col_1) = \genpmat{1\\0\\0}\]

Let's set up some arbitrary A

\[A = \genpmat{1&2&3\\4&5&6\\7&8&9}\]

\[
	\genpmat{
		&&\bigm|&1\\
		&A&\bigm|&0\\
		&&\bigm|&0		
	}
	\to
	\genpmat{
		1&0&0&\bigm|&\\
		0&1&0&\bigm|&A^{-1}_1\\
		0&0&1&\bigm|&		
	}
\]

To explain the weird notation here, $A^{-1}_1$ is the first column of the inverse matrix. 

We do this same process and multiply the original matrix of A by each column of its inverse, which gets us each column of the identity matrix!

So\ldots

We can take A, place the first column as the 

\vfill
\hrule
\begin{example}
Use the above method to find the inverse of $A=\left( \begin{array}{ccc}
1 & 2&0\\
1 & 0& 2 \\
1& 1 &0
\end{array} \right)$. Check your answer!
\end{example}
\vfill

\pagebreak

\begin{center}
\underline{IV. The 2D Case}
\end{center}

Find the inverse of the general 2D matrix $A=\left( \begin{array}{cc}
a &b\\
c & d \\
\end{array} \right)$:
\vfill
\hrule
\begin{example}
Use the formula derived above to find the inverse of the matrix $A=\left( \begin{array}{cc}
1 & 2\\
2 & 3
\end{array} \right)$
\end{example}
\vskip 1truein
\hrule
What is the connection between matrices $A$ that have no inverses and the system $AX=B$?

\vfill

\end{document}
