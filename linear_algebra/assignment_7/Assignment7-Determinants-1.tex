\documentclass[11pt]{article}     \parskip=5pt
\parindent=0pt

\addtolength{\oddsidemargin}{-.925in}
\addtolength{\evensidemargin}{-.925in}
\addtolength{\textwidth}{2in}

\usepackage{graphicx} 
\usepackage{amssymb}
\usepackage{amsmath} 
\usepackage{amsfonts}
\usepackage{amsthm}
\usepackage{amstext}
%\usepackage{mathtools}
\usepackage{epstopdf}
\usepackage{multicol}
\usepackage{cancel}
\usepackage{wrapfig}
\usepackage{float}
\usepackage{enumitem}  
\usepackage{enumitem}
\usepackage{xcolor}
\usepackage{ulem}

\usepackage{capt-of}

\renewcommand{\baselinestretch}{1.1}

\newcommand{\mz}{\mathbf{Z}}
\newcommand{\mq}{\mathbf{Q}}
\newcommand{\mr}{\mathbf{R}}
\newcommand{\mC}{\mathbf{C}}
\newcommand{\mP}{\mathbf{P}}

\newcommand{\vect}{\vec}
\newcommand{\ds}{\displaystyle}


\newcommand{\mc}{\mathbf{c}}
\newcommand{\lra}{\longrightarrow}

\newcommand{\mb}{\mathbf{b}}
\newcommand{\mv}{\mathbf{v}}
\newcommand{\mU}{\mathbf{u}}
\newcommand{\mw}{\mathbf{w}}
\newcommand{\ma}{\mathbf{a}}
\newcommand{\mx}{\mathbf{x}}
\newcommand{\my}{\mathbf{y}}
\newcommand{\mo}{\mathbf{0}}


\usepackage{fancyhdr}

\pagestyle{fancy} \fancyhead[RO, RE] {\texttt{Linear Algebra - Homework 7, due Thursday, November 3}}

\begin{document}

This assignment is due either in paper format (by class time on Wed.), OR in Canvas (by \textbf{Thur. at noon}).  
If you have any questions, please ask me via email, or in my office hours (posted on Canvas).

\hrule

\begin{enumerate}
\item Evaluate the following determinants using the definition, then check your answer with your calculator: 

\begin{enumerate}
\begin{multicols}{2}
\item 
$\begin{vmatrix}
3& 0 \\
2& 3
\end{vmatrix}=$
\vskip .2truein
\item 
$\begin{vmatrix}
0& 4& 1\\
5 & -3& 0\\
2 & 3 &1
\end{vmatrix}=$

\item 
$\begin{vmatrix}
-1& 1 \\
 -3& -2
\end{vmatrix}=$
\vskip .2truein
\item 
$\begin{vmatrix}
2 & 3 & -3 \\
4 & 0 & 3\\
6 & 1 & 5
\end{vmatrix}=$
\end{multicols}
\end{enumerate}
\item Evaluate the following determinants using the minor expansion method. Be careful to mark what row or column you are expanding by: 
\begin{enumerate}
\begin{multicols}{2}

\item 
$\begin{vmatrix}
4 & 0 & 0 & 5\\
1 & 7 & 2 & -5\\
3 & 0 & 0 & 0\\
8 & 3 & 1 & 7
\end{vmatrix}=$

\item 
$\begin{vmatrix}
 3 & 0 & 0 & 0\\
 7 & -2 & 0 & 0\\
 2 & 6 & 3 & 0\\
 3 & -8 & 4 & -3
\end{vmatrix}=$
\end{multicols}
\end{enumerate}

\item Find the determinants for the following transformation matrices: 
\begin{enumerate}
\item  $\ds M=
\begin{pmatrix}
\sqrt{2}/2 & - \sqrt{2}/2 \\
\sqrt{2}/2 & \sqrt{2}/2
\end{pmatrix}$, the transformation that rotates the plane by $\pi/4$ degrees. 

\item $\ds M=
\begin{pmatrix}
\cos(\theta) & -\sin(\theta)\\
\sin(\theta) & \cos(\theta)
\end{pmatrix}$, the transformation that rotates the plane by $\theta$ degrees. 
\item Using your answers above, what can you say about rotation transformations? 
\end{enumerate}
\item Find the determinants for the following transformation matrices: 
\begin{enumerate}\item $\ds M=
\begin{pmatrix}
1/5 & 2/5\\
2/5 & 4/5
\end{pmatrix}$, the transformation that projects the plane onto the $y=2x$ line. 
\item $M=
\begin{pmatrix}
\ds \frac{1}{1+m^2} &\ds \frac{m}{1+m^2}\\
\ds \frac{m}{1+m^2}& \ds \frac{m^2}{1+m^2}
\end{pmatrix}$, the transformation that projects the plane onto the $y=mx$ line. 
\item Using your answers above, what can you say about projection transformations? 
\end{enumerate} 

\item Find the parameter $t$ such that the matrix $A=\begin{pmatrix}
4 & 1 & 2\\
4 & 0 & 3\\
4 & 2 & t
\end{pmatrix}$ 
is not invertible.


\item Decide if the following sets of vectors are linearly independent (show all the work, and justify your argument). You can compute the determinants with the calculator, but my advice is to compute them by hand, for more practice: 

\begin{enumerate}
\item 
$\begin{pmatrix}
 4\\ 6 \\ 2
\end{pmatrix}$ 
, $\begin{pmatrix}
7\\ 0 \\ -7
\end{pmatrix}$ 
, $\begin{pmatrix}
 -3\\ -5 \\ -2
\end{pmatrix}$ 


\item 
$\begin{pmatrix}
3 \\ 5 \\ -6 \\ 4
\end{pmatrix}$ 
, $\begin{pmatrix}
2 \\ -6 \\ 0 \\ 7
\end{pmatrix}$ 
, $\begin{pmatrix}
-2 \\ -1 \\ 3 \\ 0
\end{pmatrix}$ 
, $\begin{pmatrix}
0 \\ 0 \\ 0 \\ -2
\end{pmatrix}$


\end{enumerate}
\item Let $A$ be an $n\times n$ matrix. What is the relation between $\det(A)$ and $\det(-A)$? (Note that your answer will depend on $n$, my advice is to start by looking at $2\times 2$ and $3\times 3$ matrices, and use the linearity property of determinants,
\end{enumerate}

\hrule
And from our textbook, the following more theoretical problems from Section 3.1 Exercises on \textbf{page 85}:

\begin{enumerate}[resume]
\item Problem 3.2 (properties of the determinant)
\item Problem 3.5 (nilpotent matrices).
\item Problem 3.8 (the $3\times 3$ Vandermonde determinant).
\end{enumerate}

\end{document}



\pagebreak 
\item Use determinants to find the area of the parallelogram with vertices at $(0,0), (-2,4), (4,-5), (2,-1) $. (Hint: you may want to draw these points first, to see what vectors you want to choose.)

\item Use determinants to find the volume of the parallelepiped  with one vertex at the origin, and adjacent vertices at $A(1,0,-3), B(1,2,4),C (5,1,0)$. (Hint: for a volume computations you need the three vectors that generate this object; the best approach is to consider the vectors starting at the origin, and ending at the points $A,B,C$.)





\end{enumerate}



\end{document}

