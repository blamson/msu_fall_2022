\section*{Exercise 3.2}

How are the determinants of the given matrices related?

\subsection*{Part A}

\[
	\begin{aligned}
		A &= 
		\genericmat{p}{a_1 & a_2 & a_3 \\ b_1 & b_2 & b_3 \\ c_1 & c_2 & c_3},
		\\
		B &=
		\genericmat{p}{2a_1 & 3a_2 & 5a_3 \\ 2b_1 & 3b_2 & 5b_3 \\ 2c_1 & 3c_2 & 5c_3}
	\end{aligned}
\]

The determinant here will be multiplied by $2 \cdot 3 \cdot 5$. We can think of this as scaling the length of this object by 2, the width by 3 and the height by 5. 

As such we can state:

\[
	\begin{aligned}
		det(B) = 2 \cdot 3 \cdot 5 \cdot det(A)
	\end{aligned}
\]

\subsection*{Part B}

\[
	\begin{aligned}
		A &= \genericmat{p}{a_1 & a_2 & a_3 \\ b_1 & b_2 & b_3 \\ c_1 & c_2 & c_3},
		\\
		B &= \genericmat{p}{3a_1 & 4a_2 + 5a_1 & 5a_3 \\ 3b_1 & 4b_2 + 5b_1 & 5b_3 \\ 3c_1 & 4c_2 + 5c_1 & 5c_3}
	\end{aligned}
\]

Here we can cite Proposition 3.2 from the book.

\subsubsection{Proposition 3.2}

\emph{The determinant does not change if we add to a column a linear combination of the other columns. In particular, the determinant is preserved under column replacement.}

As such, we need only concern ourselves with the coefficients in front of the primary column variables $\left( a_2, b_2, c_2 \right)$, which, in our case, are all 4.

So, from this we can state that:

\[
	det(B) = 3 \cdot 4 \cdot 5 det(A)
\]
