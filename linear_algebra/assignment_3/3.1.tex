\section*{Exercise 3.1}

If $A$ is an $n \times n$ matrix, how are the determinants det $A$ and det$(5A)$ related?

\textbf{Note:} $det(5A) = 5 det A$ only in the trivial case of $1 \times 1$ matrices.

I'm fairly certain the determinant increases here by $5^n$. This is a drastic inflation in volume as every element is increased by 5. Let us use a standard $3 \times 3$ matrix here. 

\[\genericmat{p}{1&0&0\\0&1&0\\0&0&1}\]

The determinant of this is equal to 1 which makes intuitive sense. If we instead took the determinant of:

\[\genericmat{p}{5&0&0\\0&5&0\\0&0&5}\]

The determinant of this (calculated using the programming language R) is $125 = 5^3$.

As such, we can say that 

\[\det(5A) = \det A \cdot 5^n\]

in any $n \times n$ matrix.

\textbf{Note:} This also follows from property 12 in the textbook.

\subsubsection{Property 12}

If A is an $n \times n$ matrix, the $det(\alpha A) = \alpha^n \cdot det(A)$
