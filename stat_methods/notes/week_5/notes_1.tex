\section*{Equivalency Between CIs and Hypotheses Tests}

CIs can be used to test hypotheses. 

A CI for a parameter $\theta$ with level of confidence $100(1 - \alpha)\%$ can be used to test the hypotheses

\[
	\begin{aligned}
		H_0: \theta &= \theta_0 \\
		H_a: \theta &\neq \theta_0
	\end{aligned}
\]

with significance level $\alpha$ by invoking the following decision rule:

Reject $H_0$ if the ci doesn't contain $\theta_0$

Fail to reject $H_0$ if it does contain $\theta_0$

The CI approach will always reach the same conclusion as the associated hypothesis test.

\subsection{Example}

For a test of 

\[
	\begin{aligned}
		H_0: \mu &= 50 \\
		H_a: \mu &\neq 50
	\end{aligned}
\]

The p-value for this is 0.16. Thus, $H_0$ would not be rejected at either the 5\% or the 10\% significance levels.

\textbf{Question:} Would a 95\% CI for $\mu$ contain the value 50?

\textbf{Answer} Yes. Since we failed reject the null under an $\alpha = 0.05$, that means that 50 is contained in the confidence interval.
